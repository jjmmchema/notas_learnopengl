\newcommand{\ctheta}{\cos{\theta}}
\newcommand{\stheta}{\sin{\theta}}
\newcommand{\p}{\cdot}

\maketitle

\section{Matrices de rotación.}
    Las matrices a continuación son para transformar
vectores en 4D en los que la 4ta componente corresponde
a la anchura del vector y no a un nuevo eje, razón por la
cual se deja igual dicha componente.\par
    En otras palabras, estas matrices permiten realizar
rotaciones a vectores 3D alrededor del eje que se desee.\par

Matriz de rotación alrededor del eje x:
\begin{align*}
    \begin{bmatrix}
        1 & 0 & 0 & 0 \\
        0 & \ctheta & -\stheta & 0 \\
        0 & \stheta & \ctheta & 0 \\
        0 & 0 & 0 & 1 \\
    \end{bmatrix} \cdot
    \begin{pmatrix} x \\ y \\ z \\ 1 \end{pmatrix} = 
    \begin{pmatrix} x \\
                    \ctheta \p y - \stheta \p z \\
                    \stheta \p y + \ctheta \p z \\ 
                    1\end{pmatrix}
\end{align*}

Matriz de rotación alrededor del eje y:
\begin{align*}
    \begin{bmatrix}
        \ctheta & 0 & \stheta & 0 \\
        0 & 1 & 0 & 0 \\
        -\stheta & 0 & \ctheta & 0 \\
        0 & 0 & 0 & 1 \\
    \end{bmatrix} \cdot
    \begin{pmatrix} x \\ y \\ z \\ 1 \end{pmatrix} = 
    \begin{pmatrix} \ctheta \p x + \stheta \p z \\
                    y \\
                    -\stheta \p x + \ctheta \p z \\ 
                    1\end{pmatrix}
\end{align*}

Matriz de rotación alrededor del eje z:
\begin{align*}
    \begin{bmatrix}
        \ctheta & -\stheta & 0 & 0 \\
        \stheta & \ctheta & 0 & 0 \\
        0 & 0 & 1 & 0 \\
        0 & 0 & 0 & 1 \\
    \end{bmatrix} \cdot 
    \begin{pmatrix} x \\ y \\ z \\ 1 \end{pmatrix} = \begin{pmatrix} \ctheta \p x - \stheta \p y \\
                    \stheta \p x + \ctheta \p y \\
                    z \\ 1\end{pmatrix}
\end{align*}

\section{Combinación de matrices de transformación y escalado.}
    Es importante saber que se pueden combinar las matrices
de transformación, rotación y escalado para no tener que
aplicarlas individualmente a un vector sino en conjunto como
una sola matriz.\par
    Como la multiplicación de matrices no es commutativa,
se debe primero hacer el escalado, luego la rotación y por
último las transformaciones para obtener los resultados deseados. En otras palabras:\par
\begin{align*}
    \begin{bmatrix} \textit{Transformación} \end{bmatrix} \p
    \begin{bmatrix} \textit{Escalado} \end{bmatrix} =
    \begin{bmatrix} \textit{Combinación} \end{bmatrix}
\end{align*}
